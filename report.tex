\documentclass[a4paper,10pt]{article}
\usepackage{multicol}
\usepackage{listings}
\author{Koos van Strien (5783437), Tim van Deurzen ()}
\title{Verslag}

\begin{document}

\maketitle

\section{Code representatie}
De code is in verschillende lagen onder te verdelen.

\subsection{Het bestand}
Het bestand waar de code zich in bevindt is de hoogste ``laag'' die binnen de
compiler als geheel gezien kan worden.  Aangezien een bestand al een geheel is,
zou hier geen representatie van hoeven zijn binnen de optimizer. Aangezien het
bestand echter ook verzameling en opslagplaats is van basic blocks (de
eerstvolgende laag), bestaat binnen de optimizer de klasse ``blockBuilder'' die
codebestanden representeert. Binnen de klasse bevinden zich methoden om basic
blocks te herkennen en te achterhalen wat de targets zijn waar vanaf een basic
block naartoe gesprongen wordt.


\subsection{De basic blocks}
Binnen het bestand is de code onder te verdelen in basic blocks. Basic blocks
beginnen met een mogelijke jump-target en eindigen met een jump of branch
operatie. Binnen de optimizer worden deze gerepresenteerd door instanties 
van de ``basicBlock'' klasse. Binnen deze klasse bevinden zich methoden om
informatie over basic blocks te verkrijgen en om veranderingen aan te brengen
op basic block-niveau.

\subsection{Operations en labels}
\subsubsection{Categorie\"en}
Het laagste niveau zijn de afzonderlijke labels en operations. Binnen de
optimizer zijn deze instructies in zes categorie\"en verdeeld. Samen met de
categorie voor labels valt elke regel code in \'e\'en van de volgende
categorie\"en:
\begin{itemize}
\item Control
\item Load
\item Store
\item Integer Arithmetic
\item Float Arithmetic
\item System
\item Label
\end{itemize}

\subsubsection{Klassen}
Elke categorie heeft een eigen klasse. Boven alle klassen staat de
parent-klasse ``operation''. Generieke methoden worden in deze klasse
ge\"implementeerd, specifiekere methoden worden overridden of gespecifieerd
binnen de kind-klassen. Naast de ``operation'' klasse hebben de klassen voor de
categorie\"en Load en Store nog een gezamelijke parent-klasse, aangezien zij
sommige acties delen. Binnen de ``operation''-klasse is een static methode die
bij aanroep automatisch een instantie van de juiste klasse teruggeeft.

\subsubsection{Bewerkingen op code}
Wijzigingen aan operations worden altijd via methoden gedaan. Aangezien elke
operation door een instantiatie wordt vertegenwoordigd, is dit een vrij
intu\"itieve methode. ((NIET ZEKER OF DIT ERIN MOET: Het is ook een veilige
methode, aangezien de mogelijkheden om wijzigingen aan te brengen getest kunnen
worden en daarna ``gegarandeerd'' werken.))

\section{Optimalisatie}
De optimizer heeft de volgende optimalisaties ge\"implementeerd:
\begin{itemize}
\item Redundant labels
\item Redundant load / stores
\item Copy propagation
\item Common subexpression elimination
\end{itemize}

\subsection{Structuur}
Er is een parent-klasse ``optimizationClass''. Deze voorziet in de generieke
functionaliteit die elke optimalisatie nodig heeft, zoals het opslaan en
weergeven van basic blocks. 

\subsection{Redundant labels}
\subsubsection{Doel}
De redundant labels-optimalisatie kijkt of een basic block a als enige
operatie een jump heeft (naar basic block b). In dat geval wordt geheel basic
block a verwijderd. Verwijzingen naar basic block a worden vervangen door
verwijzingen naar basic block b. Dit scheelt dan een jump-operatie.

\subsubsection{Implementatie}
Het algoritme ge\"implementeerd volgens onderstaand pseudocode. Wanneer er over
een ``target'' van een block gesproken wordt wordt hier altijd een block
bedoelt waar een jump-operatie naar toe springt.
\begin{verbatim}
for each basic block __b:
  if __b has a target __t and __t has exactly two operations 

    if the last operation of __t is a CONTROL:
      save the target of __t as __l
    
    if __l is a label and the last operation __o of __b is a CONTROL:
      set the target of __o to __l

      exclude __t from the code
\end{verbatim}

\subsection{Redundant load / stores}
\subsubsection{Doel}
Redundant load / store optimalisatie heeft als doel geen waarde uit het
geheugen te laden die zich al in een register bevindt (redundant load), en geen
waarde terug te schrijven naar het geheugen wanneer deze niet gewijzigd is
(redundant store). Aangezien load- en store-operaties door de writeback erg
traag zijn, kan dit al snel uitvoertijd schelen, met name wanneer bijvoorbeeld
binnen een loop hierdoor geheugen-operaties weggelaten kunnen worden.

Ook staan operaties soms achter een load gescheduled omdat de waarde daarvoor
nog niet beschikbaar zou zijn. Wanneer deze load redundant blijkt te zijn is er
voor niets gewacht met het uitvoeren van de instructie.

\subsubsection{Implementatie}
Het algoritme is als volgt ge\"implementeerd:
\begin{verbatim}
for each basic block __b:
  clear list of operations that change a register

  for each operation __op in __b:
    if __op was previously excluded (by another optimized):
      skip operation
    else:
      search for an operation, __op2, in the list that uses the same
      registers as __op

      if __op is a LOAD:
        if __op2 exists:
          if __op2 is a load or a store:
            if the target of __op2 is the target of __op and the address of
              __op2 is the address of __op:

              exclude __op from the code.

            else if the targets of __op2 and __op are the same but the
              addresses are not:

              remove __op2 from the list.
              add __op to the list.
          else:
            if the target of __op2 and __op are the same:

                remove __op2 from the list.
                add __op to the list.
        else if __op was previously stored:
          exclude __op from the code.

        else:
          add __op to the list of operations

      else if __op is a STORE:
        if __op was previously stored:
          exclude __op from the code.
        else:
          if __op2 exists:
            remove __op2 from the list of operations.
          else:
            add __op to the list of operations.

      else if __op is an ARITHMETIC operation:
        if __op2 exists:
          remove __op2 from the list/

        add __op to the list of operations.
\end{verbatim}

\subsection{Copy propagation}
\subsubsection{Doel}
Copy propagation zou ook omschreven kunnen worden als "redundant moves
elimination". Het gaat hier namelijk om move-instructie die een register
kopi\"eren, maar waar het oude register niet overschreven wordt voordat een
operatie het nieuwe register uitleest. Het vervangen van het ``doelregister''
door het ``bronregister'' heet copy propagation. Wanneer uiteindelijk het
doelregister niet meer gebruikt wordt, of een nieuwe waarde krijgt, terwijl het
bronregister tot de laatste aanroep aan toe beschikbaar was, kan de gehele
move-operatie verwijderd worden.

\subsubsection{Implementatie}

\begin{verbatim}
for each basic block __b:
  clear list of moves
  clear list of dead_moves
  clear list of updatedOperations

  for each operation __op in __b:
    if __op was previously excluded (by another optimizer):
      skip operation
    else:
      if __op is a MOVE:
        if __op uses $fp / $sp:
          skip operation

        if there is a reverse move on list:
          remove old operation

        add __op to list of moves


      if __op is a STORE:
        if there is a move in the list of moves that has a target that __op
          uses as source:

          update __op and set the target to the source of the move.
          exclude the move from the code.

          add __op and move to a list of altered operations


      if __op is a LOAD:
        if there is a move in the list of moves that has the same target as
          __op or __op uses the source of the move as target:

          remove move from list
          add move to list of dead_moves

        if the source or destination of the move is used as an offset in __op:

          rollback all edits until the last added move on the list of
          altered operations.

        if there is a move in the list of dead moves:
          if the destination of the move is the target of the load or
            the source of the move is the target of the load:

            remove move from list of dead_moves

          if the source or destination of the move is used as an offset in __op:

            rollback all edits until the last added move on the list of
            altered operations.

      
      if __op is an ARITHMETIC or CONTROL operation:
        if there is a move in the list of moves that has a target that is
          the a source of __op:

          update __op by replacing the old register with the source of the
          move.

          exclude the move from the code.

          add __op and move to the list of altered operations.

        else if there is a move in the list of dead moves:
          if the destination or the source of the move is the target
            of __op:

            remove the move from the list of dead moves.
\end{verbatim}

\subsection{Common subexpression elimination}
\subsubsection{Doel}
Wanneer dezelfde berekening meerdere keren uitgevoerd wordt (met dezelfde
argumenten, dus dezelfde uitkomst) heet dit een common subexpression. De 
tweede berekening kan dan vervangen worden door een move-instructie, of,
wanneer zowel de operatie als de parameters hetzelfde zijn, de hele operatie
verwijderd worden.

In beginsel lijkt het eerste een vrij zinloze optimalisatie, aangezien er geen
direct resultaat is. Een move-operatie is namelijk even duur als een
berekening. Echter, het impliceert wel dat registers eerder vrij komen, wat
weer kan resulteren in nieuwe redundant stores (die met move-operaties
eenvoudig te herkennen zijn), copy propagation en vervolgens dead code
elimination.

\subsubsection{Implementatie}
In de optimizer is alleen het tweede deel van de common subexpression
elimination ge\"implementeerd. Houd bij het lezen van de pseudocode in
gedachten dat er per block een lijst wordt bijgehouden met op dit moment
aanwezige expressions.
\begin{verbatim}
for each basic block in __b:
  for each operation __op in __b:
    if __op was previously excluded (by another optimizer):
      skip operation
    else:

      if __op has arguments:
        look in the list of expressions already passed by if current
          expression is updating an earlier used one.
      else:
        current operation is no expression

      if __op is of an INT_ARITHMETIC or FLOAT_ARITHMETIC
        commonExpression = self.findCommonExpression(__op )
        
        if __op has a common expression
          exclude __op from block
        else if __op is not updating another expression:
          add __op to the list of expressions

      if __op is a LOAD and op is updating an expression:
        remove the expression __op is updating from the list
\end{verbatim}

\section{Resultaten}
De versnellingen die er gerealiseerd zijn:
\begin{itemize}
    \item acron | van 4436540 naar 4360426 cycles | 2.8\%
    \item clinpack | van 1556370 naar 60067 cycles | inconsistent
    \item pi | van 1715292 naar 1708329 cycles | 2\%
    \item whet | van 2864289 naar 2970553 cycles | 4.6\%
\end{itemize}

\end{document}
